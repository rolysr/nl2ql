\begin{abstract}
	Esta tesis se centra en abordar la complejidad inherente a la consulta de bases de datos en forma de grafo, como Neo4J. Estas bases de datos a menudo requieren un conocimiento especializado en lenguajes de consulta, lo que limita su accesibilidad a un grupo reducido de usuarios con habilidades técnicas avanzadas. Para superar esta limitación, proponemos la aplicación del aprendizaje \textit{Zero-Shot}, un enfoque innovador en el procesamiento del lenguaje natural. En esta investigación, se lleva a cabo un experimento basado en el modelo \texttt{GPT-4} para traducir consultas de lenguaje natural a código \textit{Cypher}. La evaluación se realiza utilizando el conjunto de datos de evaluación \texttt{MetaQA}, que abarca una amplia variedad de ejemplos de consultas. Los resultados obtenidos fueron del $76.53\%$, $43.45\%$ y $31.03\%$ para los tres lotes de evaluación del \textit{benchmark} utilizado, mejorando de esta forma el mejor resultado de modelos de lenguaje en la traducción de lenguaje natural a código \textit{Cypher} sobre \texttt{MetaQA} mediante el aprendizaje \textit{Zero-Shot}.
\end{abstract}

\begin{enabstract}
	This thesis focuses on addressing the inherent complexity of querying graph databases, such as Neo4J. These databases often require specialized knowledge in query languages, limiting their accessibility to a small group of users with advanced technical skills. To overcome this limitation, we propose the application of Zero-Shot learning, an innovative approach in natural language processing. In this research, an experiment is conducted based on the GPT-4 model to translate natural language queries into Cypher code. The evaluation is carried out using the MetaQA evaluation dataset, which covers a wide variety of query examples. The results obtained were $76.53\%$, $43.45\%$, and $31.03\%$ for the three evaluation lots of the benchmark used, thereby improving the best result of language models in translating natural language into Cypher code using Zero-Shot learning.
\end{enabstract}