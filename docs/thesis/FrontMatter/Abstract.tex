\begin{abstract}
	Esta tesis se centra en abordar la complejidad inherente a la consulta de bases de datos en forma de grafo, como Neo4J. Estas bases de datos a menudo requieren un conocimiento especializado en lenguajes de consulta, lo que limita su accesibilidad a un grupo reducido de usuarios con habilidades técnicas avanzadas. Para superar esta limitación, proponemos la aplicación del aprendizaje Zero-Shot, un enfoque innovador en el procesamiento del lenguaje natural. En esta investigación, se lleva a cabo un experimento basado en el modelo <variable> para traducir consultas de lenguaje natural a código \textit{Cypher}. La evaluación se realiza utilizando el conjunto de datos de evaluación <variable>, que abarca una amplia variedad de ejemplos de consultas. Los resultados obtenidos, <variable>, establecen un punto de referencia esencial para el uso de modelos de lenguaje en la traducción de lenguaje natural a código \textit{Cypher}.
\end{abstract}

\begin{enabstract}
	This thesis focuses on addressing the inherent complexity of querying graph databases like Neo4J. Such databases often require specialized knowledge of query languages, limiting accessibility to a small group of users with advanced technical skills. To overcome this limitation, we propose the application of Zero-Shot learning, an innovative approach in natural language processing. In this research, an experiment is conducted based on the <variable> model to translate natural language queries into \textit{Cypher} code. Evaluation is carried out using the <variable> evaluation dataset, which encompasses a wide variety of query examples. The obtained results, <variable>, establish a crucial benchmark for the use of language models in translating natural language to \textit{Cypher} code.
\end{enabstract}