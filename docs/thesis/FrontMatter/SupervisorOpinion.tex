\begin{opinion}	
	

\vspace{1cm}


\begin{flushright}
	\underline{\hspace{6.5cm}}\\
	Dr. Alejandro Piad Morffis
	
	Facultad de Matemática y Computación
	
	Universidad de la Habana
	
	Noviembre, 2023
\end{flushright}

Opinión del Tutor para la tesis de Licenciatura en Ciencia de la Computación ``Enfoques Zero-Shot para la Extracción de Conocimiento a partir de Lenguaje Natural" de Rolando Sánchez Ramos:

El tema de la tesis aborda una motivación fundamental en el campo de la Ciencia de la Computación, dado que la existencia de conocimiento estructurado en bases de datos en forma de grafos ha cobrado gran relevancia en la actualidad. A medida que se ha observado un creciente uso de modelos de lenguajes para el descubrimiento de conocimientos, se ha vuelto claro que estos modelos enfrentan desafíos significativos al integrar conocimiento estructurado. Los problemas que presentan al alucinar y la dificultad para realizar razonamientos que involucren varios pasos de inferencia representan un obstáculo fundamental que merece una solución innovadora.

La propuesta de solución presentada por el estudiante Rolando Sánchez Ramos consiste en utilizar un modelo de lenguaje para generar una consulta en un lenguaje formal intermedio, que pueda ser ejecutada en una base de conocimientos estructurada. Posteriormente, la respuesta generada representa los datos que corresponden a la consulta inicial en lenguaje natural. Esta aproximación innovadora y técnica, plantea una solución prometedora para integrar eficazmente el conocimiento estructurado con los modelos de lenguajes, superando así los desafíos mencionados.

Es importante destacar el esfuerzo y la capacidad de Rolando Sánchez Ramos para abordar de manera independiente un tema de gran relevancia y complejidad en el área de la Ciencia de la Computación. Su habilidad técnica para implementar algoritmos que involucren sistemas de \textit{Machine Learning} y sistemas tradicionales de bases de datos es impresionante y demuestra un alto nivel de conocimiento y destreza en este campo.

Esperamos que la tesis de Rolando Sánchez Ramos reciba la evaluación máxima que merece, ya que constituye una contribución significativa al campo. Agradecemos al estudiante por su arduo trabajo y por su valioso aporte a la comunidad académica en el área de la Ciencia de la Computación.

\end{opinion}