%===================================================================================
% Chapter: Introduction
%===================================================================================
\chapter*{Introducción}\label{chapter:introduction}
\addcontentsline{toc}{chapter}{Introducción}
%===================================================================================

\qquad 

Con el aumento gradual de la información en la actualidad, el proceso de organización y extracción de conocimiento en base a esta se ha convertido en una tarea fundamental. La razón principal, es la necesidad de almacenar dichos datos con el objetivo de ser consultados en un futuro de forma eficiente. Por lo tanto, para llevar a cabo dicho reto, ha sido imprescindible el desarrollo de sistemas capaces de persistir información de forma estructurada y facilitar el acceso a esta.

Las bases de datos orientadas a grafos \cite{graphdbs} constituyen herramientas que permiten el almacenamiento y consulta de información de manera escalable y segura. Un ejemplo de estas es \textit{Neo4J} \cite{neo4j}, con la cual se puede interactuar a partir del lenguaje de programación \textit{Cypher} \cite{cypher}.

Para utilizar el lenguaje de consulta \textit{Cypher} se requiere de conocimientos básicos de programación, lo cual consume cierto tiempo y esfuerzo. Esto tiene como consecuencia que, solo aquellas personas con experiencia en el uso de lenguajes de programación puedan hacer uso de la mayoría de los sistemas de almacenamiento de datos. Por lo tanto, es necesaria una herramienta que permita democratizar dicho proceso, para lo cual se propone un modelo capaz de traducir una consulta en lenguaje natural a un código en \textit{Cypher}. Además, también es objetivo de este trabajo experimentar con los límites del aprendizaje \textit{Zero-Shot} \cite{zeroshot} para dicha tarea. 

\subsection*{Objetivos}
<Objetivos generales>

Para lograr los objetivos generales se trazaron los siguientes objetivos específicos:

\begin{enumerate}
	\item Objetivo 1.
\end{enumerate}


[Hablar sobre la estructuracion del documento]



