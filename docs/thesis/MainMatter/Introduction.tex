%===================================================================================
% Chapter: Introduction
%===================================================================================
\chapter*{Introducción}\label{chapter:introduction}
\addcontentsline{toc}{chapter}{Introducción}
%===================================================================================

\qquad 

En la época actual, asistimos a un constante aumento en la producción de información en diversos formatos: visual, auditivo y textual, que abarca todos los ámbitos de la sociedad \cite{datagenworld}. De manera particular, resulta sumamente intrigante la información generada a través del ingenio creativo y la investigación humana. Estos tipos de datos se almacenan debido a su relevancia y a la necesidad de acceder a ellos en el futuro, pudiendo optar por una organización estructurada o no. Sorprendentemente, solo alrededor del 20\% de la información a nivel mundial se encuentra estructurada \cite{structdata}.

Las bases de conocimiento constituyen un tipo particular de bases de datos diseñadas para la administración del saber. Estas bases brindan los medios para recolectar, organizar y recuperar digitalmente un conjunto de conocimientos, ideas, conceptos o datos \cite{orgkb}. La ventaja fundamental de mantener la información de manera estructurada radica en su facilidad para ser consultada, ampliada y modificada. Debido a su utilidad y prevalencia, la recuperación de información a través de consultas en bases de conocimiento se ha convertido en una tarea esencial.

Es esencial que la información almacenada en bases de conocimiento adopte un formato adecuado para permitir búsquedas ágiles y precisas. Entre los formatos más comunes se encuentran los modelos de Entidad-Relación y el modelo Relacional. A pesar de ser enfoques más antiguos, el modelo Relacional (BDR) sigue siendo el más ampliamente utilizado en la actualidad \cite{datamodel}. No obstante, en ocasiones, las características específicas del problema demandan un formato más expresivo, y es en este punto donde las bases de datos orientadas a grafos (BDOG) \cite{graphdbs} entran en juego.

Las BDOG han ganado progresivamente popularidad como una manera efectiva de almacenar información en los últimos años. Estas bases tienen la capacidad de modelar una diversidad de situaciones del mundo real al tiempo que mantienen un alto nivel de simplicidad y legibilidad para los seres humanos. Las BDOG presentan numerosas ventajas en comparación con las bases de datos relacionales. Esto incluye un mejor rendimiento, permitiendo el manejo más rápido y eficaz de grandes volúmenes de datos relacionados; flexibilidad, ya que la teoría de grafos en la que se basan las BDOG permite abordar diversos problemas y encontrar soluciones óptimas; y escalabilidad, ya que las bases de datos orientadas a grafos permiten una escalabilidad eficaz al facilitar la incorporación de nuevos nodos y relaciones entre ellos. Ejemplo de un sistema de gestión de BDOG es \textit{Neo4J} \cite{neo4j}, a través del cual es posible construir instancias de este tipo de base de datos e interactuar con las mismas a través del lenguaje de programación \textit{Cypher} \cite{cypher}, el cual posee una sintaxis declarativa similar a \textit{SQL} \cite{sqllang}.

Por otro lado, el avance en la comprensión del lenguaje natural se ha visto potenciado con el surgimiento de los grandes modelos de lenguajes (LLMs) \cite{llms} como GPT-4 \cite{gpt4} o LLaMA-2 \cite{llama2}, los cuales presentan una serie de habilidades emergentes como elaboración de resúmenes de textos, generación de código, razonamiento lógico, traducción lingüística entre otras \cite{llmsskills}. Dichas herramientas constituyen modelos de \textit{Machine Learning} entrenados con un gran volumen de datos, lo cual es posible gracias al número de parámetros con los que estos son configurados \cite{llmsparams}. 

Usualmente, para el uso de los LLMs basta con ofrecerles como dato de entrada un texto (\textit{prompt}), el cual describe la tarea que se espera que estos realicen. Además, son muchas las técnicas existentes para elaborar una entrada de calidad, esto con el objetivo de que la respuesta por parte de dicho modelo de lenguage ofrezca resultados alentadores al respecto, lo cual se conoce como \textit{prompt engineering} \cite{prompengineering}. Una técnica bastánte común es \textit{Zero-Shot Learning} (ZSL) \cite{zeroshotlearning}, la cual consiste en describirle a un LLM un procedimiento a realizar sin ofrecer de antemano ejemplos de cómo resolverlo, como por ejemplo, en tareas relacionadas con la generación de código, donde algunos de estos son capaces de generar algoritmos expresados en un lenguaje de programación formal a partir de una sentencia o consulta en lenguaje natural sin recibir como entrada del usuario algunos ejemplos de código, o especificaciones de cómo funciona el lenguaje objetivo a generar \cite{tex2code1} \cite{text2code2}. 

En lo que respecta a la comprensión del lenguaje natural y su uso en consultas a bases de conocimiento, existen diversas vías llevadas a cabo y con resultados diversos, donde se hacen análisis sintácticos y semánticos sobre la consulta, muchas veces asistidos por diccionarios o mapas sobre la base de conocimiento en cuestión. Se usan modelos de paráfrasis como técnica de aumento de datos y finalmente Transformers \cite{transformers} o incluso LLMs para llevar de la consulta ya curada al lenguaje de consulta formal o a un lenguaje intermedio capaz de expresar a esta a alto nivel \cite{text2sql1} \cite{text2sql2} \cite{text2cypher1} \cite{text2cypher2}.

Por las razones anteriormente expuestas, resulta interesante la investigación sobre la tarea de generación de código de consulta formal a partir de una sentencia en lenguaje natural mediante el uso de LLMs, especialmente el diseño e implementación un experimento capaz de demostrar las capacidades reales de estos para dicho acometido, lo cual designará la importancia de continuar el estudio de dichas herramientas con el objetivo de mejorar los sistemas de extracción de conocimientos en BDOG.

\subsection*{Problemática}
Para utilizar el lenguaje de consulta formal \textit{Cypher} se requiere de conocimientos básicos de programación, lo cual consume cierto tiempo y esfuerzo. Esto tiene como consecuencia que, solo aquellas personas con experiencia en el uso de lenguajes de programación puedan hacer uso de la mayoría de los sistemas de almacenamiento de datos desarrollados con esta tecnología y teniendo en cuenta la necesidad de poseer un conocimiento del dominio sobre el cual está construida la base de datos a consultar. Por lo tanto, llevar a cabo una mejora en las herramientas orientadas a democratizar dicho proceso permitiría hacer más rápido y eficiente dicho proceso de consulta en cuanto a tiempo y recursos computacionales. Debido a dicha situación, se propone una experimentación basada en un LLM capaz de traducir una consulta en lenguaje natural a un código en \textit{Cypher}, donde a su vez se verifique la efectividad de este a partir de enfoques basados en ZSL, los cuales intuitivamente pueden ofrecer como resultado una cota inferior para la efectividad de sistemas desarrollados en base a dichos algoritmos de aprendizaje. Además, actualmente la implementación de sistemas de generación de código de consulta formal está principalmente orientada al lenguaje \textit{SQL}, mientras que para el lenguaje \textit{Cypher}, no existen suficiente estudios recientes que avalen la calidad de tales herramientas para dicho caso de uso, incluso cuando las bases de datos orientadas al segundo lenguaje representan muchos de los sistemas de almacenamiento de conocimientos, como por ejemplo, las redes semánticas.

\subsection*{Objetivos}
Dadas las ideas anteriores, los objetivos principales del trabajo consistirá en diseñar e implementar una estrategia experimental capaz de evaluar la capacidad mínima de los LLMs para la consulta en lenguaje natural a bases de conocimiento estructuradas con independencia del dominio, para lo cual se empleará un enfoque basado en ZSL.

Para lograr los objetivos generales se trazaron los siguientes objetivos específicos:

\begin{enumerate}
	\item Estudiar el estado del arte de los modelos de Aprendizaje Automático capaces de hacer predicciones de tipo texto-a-texto.
	\item Analizar el trabajo de tesis sobre este tema anteriormente desarrollado en la facultad.
	\item Implementar un modelo de Aprendizaje Automático capaz de convertir una consulta en lenguaje natural humano a un lenguaje formal que permita obtener datos a partir de una 		base de conocimiento.
	\item Explorar las capacidades de enfoques Zero-Shot para la traducción de lenguaje natural al lenguaje \textit{Cypher} con el fin de desarrollar un modelo capaz de realizar dicha tarea sin necesidad de ser entenados directamente para la misma.
	\item Mejorar el sistema de evaluación de resultados permitiendo que el conjunto de datos de prueba y evaluación sea lo más realista posible y con una mayor complejidad.
\end{enumerate}

\subsection*{Organización de la tesis}

El presente documento está estructurado en 5 capítulos que engloban las etapas de cubiertas en la investigación. En el capítulo 1, Estado del Arte, se reseña el estado actual de la teoría, herramientas y técnicas más usadas en los temas tratados. En el capítulo 2, Propuesta de Solución, se propone una sistema que responde a algunas de las limitaciones principales de los modelos desarrollados en el estado del arte. Para ello propone una vía de resolver el problema basada fundamentalmente en el aprendizaje \textit{Zero-Shot}. En el capítulo 3, Detalles de Implementación, se describen por menores en la implementación del modelo propuesto como solución, se esclarecen decisiones de diseño, y se muestran porciones del código referente a los componentes principales desarrollados. En el capítulo 4, Análisis Experimental, se sugiere un marco experimental para analizar los resultados obtenidos durante la investigación y se comprueba experimentalmente una mejora con respecto a modelos utilizando enfoques similares en el estado del arte. Finalmente, se formulan las conclusiones, que recogen los resultados obtenidos en la investigación en función de los objetivos definidos, así como las recomendaciones, donde se proponen siguientes lineas de trabajo a ser exploradas en continuación de la investigación actual. Para finalizar se indican las referencias bibliográficas consultadas, con el fin de complementar la información provista en el trabajo.



