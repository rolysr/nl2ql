\documentclass[12pt,oneside]{uhthesis}
\usepackage{subfigure}
\usepackage[ruled,lined,linesnumbered,titlenumbered,algochapter,spanish,onelanguage]{algorithm2e}
\usepackage{amsmath}
\usepackage{amssymb}
\usepackage{amsbsy}
\usepackage{caption,booktabs}
\captionsetup{ justification = centering }
%\usepackage{mathpazo}
\usepackage{float}
\setlength{\marginparwidth}{2cm}
\usepackage{todonotes}
\usepackage{listings}
\usepackage{xcolor}
\usepackage{multicol}
\usepackage{graphicx}
\floatstyle{plaintop}
\restylefloat{table}
\addbibresource{Bibliography.bib}
% \setlength{\parskip}{\baselineskip}%
\renewcommand{\tablename}{Tabla}
\renewcommand{\listalgorithmcfname}{Índice de Algoritmos}
%\dontprintsemicolon
\SetAlgoNoEnd

\definecolor{codegreen}{rgb}{0,0.6,0}
\definecolor{codegray}{rgb}{0.5,0.5,0.5}
\definecolor{codepurple}{rgb}{0.58,0,0.82}
\definecolor{backcolour}{rgb}{0.95,0.95,0.92}

\lstdefinestyle{mystyle}{
    backgroundcolor=\color{backcolour},   
    commentstyle=\color{codegreen},
    keywordstyle=\color{purple},
    numberstyle=\tiny\color{codegray},
    stringstyle=\color{codepurple},
    basicstyle=\ttfamily\footnotesize,
    breakatwhitespace=false,         
    breaklines=true,                 
    captionpos=b,                    
    keepspaces=true,                 
    numbers=left,                    
    numbersep=5pt,                  
    showspaces=false,                
    showstringspaces=false,
    showtabs=false,                  
    tabsize=4
}

\lstset{style=mystyle}

\title{Enfoques Zero-Shot para la Extracción de Conocimiento a partir de Lenguaje Natural}
\author{\\\vspace{0.25cm}Rolando Sánchez Ramos}
\advisor{\\\vspace{0.25cm}Dr. Alejandro Piad Morffis}
\degree{Licenciado en Ciencia de la Computación}
\faculty{Facultad de Matemática y Computación}
\date{Fecha\\\vspace{0.25cm}\href{https://github.com/rolysr/nl2ql}{github.com/rolysr/nl2ql}}
\logo{Graphics/uhlogo}
\makenomenclature

\renewcommand{\vec}[1]{\boldsymbol{#1}}
\newcommand{\diff}[1]{\ensuremath{\mathrm{d}#1}}
\newcommand{\me}[1]{\mathrm{e}^{#1}}
\newcommand{\pf}{\mathfrak{p}}
\newcommand{\qf}{\mathfrak{q}}
%\newcommand{\kf}{\mathfrak{k}}
\newcommand{\kt}{\mathtt{k}}
\newcommand{\mf}{\mathfrak{m}}
\newcommand{\hf}{\mathfrak{h}}
\newcommand{\fac}{\mathrm{fac}}
\newcommand{\maxx}[1]{\max\left\{ #1 \right\} }
\newcommand{\minn}[1]{\min\left\{ #1 \right\} }
\newcommand{\lldpcf}{1.25}
\newcommand{\nnorm}[1]{\left\lvert #1 \right\rvert }
\renewcommand{\lstlistingname}{Ejemplo de código}
\renewcommand{\lstlistlistingname}{Ejemplos de código}

\begin{document}

\frontmatter
\maketitle

\begin{dedication}
    Dedicación
\end{dedication}
\chapter*{Agradecimientos}\label{chapter:agradecimientos}

Este trabajo constituye la culminación de una trayectoria de estudios que no hubiera sido posible sin el inmenso apoyo de familiares, amigos y colegas de profesión. Quiero en este apartado agradecer a una importante parte de ellas.

	A mis padres, sin los cuales no imagino haber llegado a esta instancia de mi vida. Siempre han estado a mi lado y me han apoyando incondicionalmente en la realización de mis sueños. También, a mis abuelos, cuyo orgullo hacia mí me ha inspirado a mejorar cada día más.

	A mi hermana Angélica y mi hermano Jose Diego, a quienes tengo siempre presentes y con los cuales quiero disfrutar y compartir cada logro que pueda alcanzar.

	A Marian, que con su amor y apoyo ha sido imprescindible para cumplir las metas que me he propuesto y cuya presencia en mi vida me hace sentir afortunado.

	A mis amigos David y Gabriel por estar a mi lado y ser fundamentales para decidirme a estudiar esta carrera. A Andry, por cada trabajo que hicimos juntos y cada vez que criticamos al Barça. A mis amigos Pablo y Abel, por todas las horas que estudiamos juntos y el gusto que daba compartir con ellos tanto dentro como fuera de la universidad. A mi amigo Leandro, por compartir la misma motivación y pasión hacia nuestra profesión. A Marcos, por los momentos compartidos y las dudas que me ayudó a resolver.

	A todos los profesores que tuve en la facultad durante mis años de estudios, en especial al profesor Alejandro Piad, por ser mi modelo a seguir como científico de la computación y como persona. También, al profesor Somoza, por sus conferencias y clases prácticas, las cuales disfruté cada segundo.




\begin{opinion}	
	

\vspace{1cm}


\begin{flushright}
	\underline{\hspace{6.5cm}}\\
	Dr. Alejandro Piad Morffis
	
	Facultad de Matemática y Computación
	
	Universidad de la Habana
	
	Noviembre, 2023
\end{flushright}

\end{opinion}
\begin{resumen}
	Resumen en español
\end{resumen}

\begin{abstract}
	Resumen en inglés
\end{abstract}
\tableofcontents
\listoffigures

\mainmatter

%===================================================================================
% Chapter: Introduction
%===================================================================================
\chapter*{Introducción}\label{chapter:introduction}
\addcontentsline{toc}{chapter}{Introducción}
%===================================================================================

\qquad 

En la época actual, asistimos a un constante aumento en la producción de información en diversos formatos: visual, auditivo y textual, que abarca todos los ámbitos de la sociedad \cite{datagenworld}. De manera particular, resulta sumamente intrigante la información generada a través del ingenio creativo y la investigación humana. Estos tipos de datos se almacenan debido a su relevancia y a la necesidad de acceder a ellos en el futuro, pudiendo optar por una organización estructurada o no. Sorprendentemente, solo alrededor del 20\% de la información a nivel mundial se encuentra estructurada \cite{structdata}.

Las bases de conocimiento constituyen un tipo particular de bases de datos diseñadas para la administración del saber. Estas bases brindan los medios para recolectar, organizar y recuperar digitalmente un conjunto de conocimientos, ideas, conceptos o datos \cite{orgkb}. La ventaja fundamental de mantener la información de manera estructurada radica en su facilidad para ser consultada, ampliada y modificada. Debido a su utilidad y prevalencia, la recuperación de información a través de consultas en bases de conocimiento se ha convertido en una tarea esencial.

Es esencial que la información almacenada en bases de conocimiento adopte un formato adecuado para permitir búsquedas ágiles y precisas. Entre los formatos más comunes se encuentran los modelos de Entidad-Relación y el modelo Relacional. A pesar de ser enfoques más antiguos, el modelo Relacional (BDR) sigue siendo el más ampliamente utilizado en la actualidad \cite{datamodel}. No obstante, en ocasiones, las características específicas del problema demandan un formato más expresivo, y es en este punto donde las bases de datos orientadas a grafos (BDOG) \cite{graphdbs} entran en juego.

Las BDOG han ganado progresivamente popularidad como una manera efectiva de almacenar información en los últimos años. Estas bases tienen la capacidad de modelar una diversidad de situaciones del mundo real al tiempo que mantienen un alto nivel de simplicidad y legibilidad para los seres humanos. Las BDOG presentan numerosas ventajas en comparación con las bases de datos relacionales. Esto incluye un mejor rendimiento, permitiendo el manejo más rápido y eficaz de grandes volúmenes de datos relacionados; flexibilidad, ya que la teoría de grafos en la que se basan las BDOG permite abordar diversos problemas y encontrar soluciones óptimas; y escalabilidad, ya que las bases de datos orientadas a grafos permiten una escalabilidad eficaz al facilitar la incorporación de nuevos nodos y relaciones entre ellos. Ejemplo de un sistema de gestión de BDOG es \textit{Neo4J} \cite{neo4j}, a través del cual es posible construir instancias de este tipo de base de datos e interactuar con las mismas a través del lenguaje de programación \textit{Cypher} \cite{cypher}, el cual posee una sintaxis declarativa similar a \textit{SQL} \cite{sqllang}.

Por otro lado, el avance en la comprensión del lenguaje natural se ha visto potenciado con el surgimiento de los grandes modelos de lenguajes (LLMs) \cite{llms} como GPT-4 \cite{gpt4} o LLaMA-2 \cite{llama2}, los cuales presentan una serie de habilidades emergentes como elaboración de resúmenes de textos, generación de código, razonamiento lógico, traducción lingüística entre otras \cite{llmsskills}. Dichas herramientas constituyen modelos de \textit{Machine Learning} entrenados con un gran volumen de datos, lo cual es posible gracias al número de parámetros con los que estos son configurados \cite{llmsparams}. 

Usualmente, para el uso de los LLMs basta con ofrecerles como dato de entrada un texto (\textit{prompt}), el cual describe la tarea que se espera que estos realicen. Además, son muchas las técnicas existentes para elaborar una entrada de calidad, esto con el objetivo de que la respuesta por parte de dicho modelo de lenguage ofrezca resultados alentadores al respecto, lo cual se conoce como \textit{prompt engineering} \cite{prompengineering}. Una técnica bastánte común es \textit{Zero-Shot Learning} (ZSL) \cite{zeroshotlearning}, la cual consiste en describirle a un LLM un procedimiento a realizar sin ofrecer de antemano ejemplos de cómo resolverlo, como por ejemplo, en tareas relacionadas con la generación de código, donde algunos de estos son capaces de generar algoritmos expresados en un lenguaje de programación formal a partir de una sentencia o consulta en lenguaje natural sin recibir como entrada del usuario algunos ejemplos de código, o especificaciones de cómo funciona el lenguaje objetivo a generar \cite{tex2code1} \cite{text2code2}. 

En lo que respecta a la comprensión del lenguaje natural y su uso en consultas a bases de conocimiento, existen diversas vías llevadas a cabo y con resultados diversos, donde se hacen análisis sintácticos y semánticos sobre la consulta, muchas veces asistidos por diccionarios o mapas sobre la base de conocimiento en cuestión. Se usan modelos de paráfrasis como técnica de aumento de datos y finalmente Transformers \cite{transformers} o incluso LLMs para llevar de la consulta ya curada al lenguaje de consulta formal o a un lenguaje intermedio capaz de expresar a esta a alto nivel \cite{text2sql1} \cite{text2sql2} \cite{text2cypher1} \cite{text2cypher2}.

Por las razones anteriormente expuestas, resulta interesante la investigación sobre la tarea de generación de código de consulta formal a partir de una sentencia en lenguaje natural mediante el uso de LLMs, especialmente el diseño e implementación un experimento capaz de demostrar las capacidades reales de estos para dicho acometido, lo cual designará la importancia de continuar el estudio de dichas herramientas con el objetivo de mejorar los sistemas de extracción de conocimientos en BDOG.

\subsection*{Problemática}
Para utilizar el lenguaje de consulta formal \textit{Cypher} se requiere de conocimientos básicos de programación, lo cual consume cierto tiempo y esfuerzo. Esto tiene como consecuencia que, solo aquellas personas con experiencia en el uso de lenguajes de programación puedan hacer uso de la mayoría de los sistemas de almacenamiento de datos desarrollados con esta tecnología y teniendo en cuenta la necesidad de poseer un conocimiento del dominio sobre el cual está construida la base de datos a consultar. Por lo tanto, llevar a cabo una mejora en las herramientas orientadas a democratizar dicho proceso permitiría hacer más rápido y eficiente dicho proceso de consulta en cuanto a tiempo y recursos computacionales. Debido a dicha situación, se propone una experimentación basada en un LLM capaz de traducir una consulta en lenguaje natural a un código en \textit{Cypher}, donde a su vez se verifique la efectividad de este a partir de enfoques basados en ZSL, los cuales intuitivamente pueden ofrecer como resultado una cota inferior para la efectividad de sistemas desarrollados en base a dichos algoritmos de aprendizaje. Además, actualmente la implementación de sistemas de generación de código de consulta formal está principalmente orientada al lenguaje \textit{SQL}, mientras que para el lenguaje \textit{Cypher}, no existen suficiente estudios recientes que avalen la calidad de tales herramientas para dicho caso de uso. 

\subsection*{Objetivos}
Dadas las ideas anteriores, los objetivos principales del trabajo consistirá en diseñar e implementar una estrategia experimental capaz de evaluar la capacidad mínima de los LLMs para la consulta en lenguaje natural a bases de conocimiento estructuradas con independencia del dominio, para lo cual se empleará un enfoque basado en ZSL.

Para lograr los objetivos generales se trazaron los siguientes objetivos específicos:

\begin{enumerate}
	\item Estudiar el estado del arte de los modelos de Aprendizaje Automático capaces de hacer predicciones de tipo texto-a-texto.
	\item Analizar el trabajo de tesis sobre este tema anteriormente desarrollado en la facultad.
	\item Implementar un modelo de Aprendizaje Automático capaz de convertir una consulta en lenguaje natural humano a un lenguaje formal que permita obtener datos a partir de una 		base de conocimiento.
	\item Explorar las capacidades de enfoques Zero-Shot para la traducción de lenguaje natural al lenguaje Cypher.
	\item Mejorar el sistema de evaluación de resultados permitiendo que el conjunto de datos de prueba y evaluación sea lo más realista posible y con una mayor complejidad.
\end{enumerate}

\subsection*{Organización de la tesis}

[Hablar sobre la estructuracion del documento]




\chapter{Estado del Arte}\label{chapter:state-of-the-art}

\chapter{Propuesta}\label{chapter:proposal}

\chapter{Detalles de Implementación y Experimentos}\label{chapter:implementation}


\backmatter

%===================================================================================
% Chapter: Conclusiones
%===================================================================================
\chapter*{Conclusiones}\label{chapter:conclusions}
\addcontentsline{toc}{chapter}{Conclusiones}
Después de aplicar \texttt{GPT-4} para la traducción de consultas al lenguaje \textit{Cypher}, este estudio presenta conclusiones relevantes tanto en términos de fortalezas como de deficiencias. Destaca la eficiencia del modelo en generar código \textit{Cypher} compilable, con un destacable $92\%$ de éxito en las pruebas, lo que subraya su competencia en la creación de consultas sin errores sintácticos o semánticos. Esta alta precisión es crucial para su aplicación práctica en bases de datos como \textit{Neo4J}.

Sin embargo, el modelo exhibe limitaciones significativas al manejar consultas más complejas. Mientras que en consultas sencillas (\textit{1-hop}) la precisión es del $76.53\%$, en consultas más complejas (\textit{2-hop} y \textit{3-hop}) esta precisión disminuye drásticamente a $43.45\%$ y $31.03\%$, respectivamente. Esto indica que aunque \texttt{GPT-4} es eficaz en traducciones simples, su rendimiento se ve comprometido en escenarios que involucran múltiples relaciones entre entidades.

Además, se identificaron varias áreas críticas no abordadas en el estudio, como la incapacidad del modelo para manejar consultas anidadas y funciones de agregación. Esto limita su utilidad en análisis de datos más complejos. La falta de un análisis multidominio también plantea preguntas sobre la capacidad de generalización del modelo, un factor esencial para determinar su eficacia en diferentes contextos y bases de datos. Otro aspecto a mejorar es el formato básico de las respuestas generadas, que no satisface necesidades de análisis de datos más detallado y avanzado. Además, se señala que el tamaño limitado del esquema de la base de datos utilizada no puso a prueba la capacidad del modelo para manejar esquemas más grandes, una limitación importante para su aplicación práctica.

A pesar de estos desafíos, el sistema de traducción propuesto supera al modelo \texttt{GPT-3} en la traducción de lenguaje natural a \textit{Cypher} en el \textit{benchmark} \texttt{MetaQA}, aunque todavía no alcanza los mejores resultados en la extracción de conocimiento usando \textit{Cypher}. 

Gracias a las medidas de precisión, recuperación y medida $F1$ utilizadas para evaluar la capacidad del modelo para la extracción de información, considerando los verdaderos positivos, falsos positivos y falsos negativos, demuestran que, a pesar de no obtener todos los resultados esperados en ciertas consultas, es importante entender la distancia entre la respuesta dada y la esperada.

Finalmente, es evidente que la eficacia del modelo disminuye con el aumento de la complejidad de las consultas, especialmente en aquellas que requieren la predicción correcta de las direcciones de relaciones entre un mayor número de entidades. Este estudio deja claro que, mientras que el uso de un Gran Modelo de Lenguaje con la técnica de aprendizaje \textit{Zero-Shot} muestra una eficiencia notable en ciertos aspectos, aún hay un camino considerable por recorrer para mejorar su rendimiento en escenarios más complejos y variados.

\chapter*{Recomendaciones}\label{chapter:conclusions}
Para futuros trabajos en la aplicación de grandes modelos de lenguajes en la traducción de consultas a \textit{Cypher}, se recomienda abordar varias áreas clave para mejorar la eficacia y versatilidad del modelo. Estas recomendaciones incluyen:

\begin{itemize}
   \item \textbf{Mejorar la Comprensión de Consultas Complejas}: Es esencial perfeccionar la capacidad del modelo para manejar consultas con múltiples relaciones entre entidades (\textit{2-hop} y \textit{3-hop}), que actualmente presentan una disminución significativa en la precisión. Esto podría implicar un entrenamiento adicional específico para estos tipos de consultas o la implementación de algoritmos más sofisticados para la comprensión de relaciones complejas.

  \item \textbf{Gestión de Consultas Anidadas y Funciones de Agregación}: Desarrollar sistemas de evaluación que contengan consultas anidadas y funciones de agregación, ampliando su aplicabilidad en análisis de datos avanzados y complejos.

   \item \textbf{Ampliación del Esquema de la Base de Datos}: Probar el modelo con esquemas de bases de datos más extensos y complejos permitiría evaluar y mejorar su capacidad de manejar casos más cercanos a escenarios del mundo real. Proponer una metodología para resolver dicho problema a partir de la adición de modulos adicionales de preprocesamiento de la consulta de entrada que permitan reducir el tamaño de la descripción de la base de datos, mostrando solamente los apectos más relevantes para la consulta en lenguaje humano a responder.

   \item \textbf{Análisis Multidominio}: Realizar pruebas en múltiples dominios y tipos de bases de datos podría ayudar a evaluar y mejorar la capacidad de generalización del modelo, lo que es crucial para su eficacia en diferentes contextos.

     \item \textbf{Mejorar del Formato de Respuestas Generadas}: Elaborar consultas de prueba que generen salidas con formatos complejos, lo cual evaluará la capacidad del sistema de devolver datos de la manera especificada.

    \item \textbf{Mejorar las estadísticas relacionadas con las consultas incorrectas durante la evaluación}: Trabajar en desarrollar algoritmos y técnicas que permitan determinar, para una salida de un gran modelo de lenguaje en la tarea de este trabajo, cuántas entidades se detectaron correctamente en el código de \textit{Cypher} generado, contabilizar y separar por grupos bien determinados las consultas de evaluación cuya respuesta falló debido a cuestiones semánticas y sintácticas, lo cual permitirá saber que estructuras del lenguaje \textit{Cypher} son inherentemente complejas de traducir.

\end{itemize}

Al implementar estas recomendaciones, futuros trabajos podrán superar las limitaciones actuales del modelo GPT-4 en la traducción de consultas a Cypher y ampliar su aplicabilidad en una variedad de entornos y situaciones prácticas.
\begin{recomendations}
    Recomendaciones
\end{recomendations}

\nocite{*}
\bibliographystyle{plain}
\bibliography{Bibliography}

\end{document}