\documentclass[11pt,oneside]{uhthesis}
%\documentclass[11pt,oneside]{report}
\usepackage{subfigure}
\usepackage[linesnumbered,lined,titlenumbered,ruled]{algorithm2e}
\usepackage{amsmath}
\usepackage{amssymb}
\usepackage{tabularx}
\usepackage{amsbsy}
\usepackage{mathpazo}
\usepackage{longtable}
\usepackage{float}
\usepackage{braket}
\setlength {\marginparwidth }{3cm}

\usepackage[spanish]{babel}
\usepackage{graphicx}

\usepackage{listings}
\usepackage{color}
\usepackage{booktabs}
\usepackage{multirow}
\usepackage{ragged2e}
\usepackage{multicol}
\usepackage{fancyhdr}

\floatstyle{ruled}
\restylefloat{table}
\usepackage{color}

\definecolor{dkgreen}{rgb}{0,0.6,0}
\definecolor{gray}{rgb}{0.2,0,0}
\definecolor{mauve}{rgb}{0.58,0,0.82}

\lstset{language=sql,
	aboveskip=10mm,
	belowskip=10mm,
	showstringspaces=false,
	columns=flexible,
	basicstyle={\small\ttfamily},
	keywordstyle=\color{blue},
	commentstyle=\color{dkgreen},
	stringstyle=\color{mauve},
	breaklines=true,
	breakatwhitespace=true,
	tabsize=3,
	numbers=left, numberstyle=\tiny, stepnumber=1,firstnumber=1,
	numbersep=5pt
}

\lstdefinelanguage{cypher}{
 morekeywords={MATCH, RETURN}}

\renewcommand{\tablename}{Tabla}
\title{Enfoques Zero-Shot para la Extracción de Conocimiento a partir de Lenguaje Natural}
\author{Rolando Sánchez Ramos}
\advisor{Dr. Alejandro Piad Morffis}
\degree{Licenciado en Ciencias de la Computación}
\faculty{Facultad de Matemática y Computación}
\date{Noviembre de 2023 \\\vspace{0.25cm}\href{https://github.com/rolysr/nl2ql}{github.com/rolysr/nl2ql}}
\logo{Graphics/uhlogo}

\renewcommand{\vec}[1]{\boldsymbol{#1}}
\newcommand{\diff}[1]{\ensuremath{\mathrm{d}#1}}

\begin{document}
\selectlanguage{spanish}

\frontmatter
\maketitle

\chapter*{Agradecimientos}\label{chapter:agradecimientos}


% Alejandro Piad Morffis




\begin{opinion}	
	

\vspace{1cm}


\begin{flushright}
	\underline{\hspace{6.5cm}}\\
	Dr. Alejandro Piad Morffis
	
	Facultad de Matemática y Computación
	
	Universidad de la Habana
	
	Noviembre, 2023
\end{flushright}

Opinión del Tutor para la tesis de Licenciatura en Ciencia de la Computación ``Enfoques Zero-Shot para la Extracción de Conocimiento a partir de Lenguaje Natural" de Rolando Sánchez Ramos:

El tema de la tesis aborda una motivación fundamental en el campo de la Ciencia de la Computación, dado que la existencia de conocimiento estructurado en bases de datos en forma de grafos ha cobrado gran relevancia en la actualidad. A medida que se ha observado un creciente uso de modelos de lenguajes para el descubrimiento de conocimientos, se ha vuelto claro que estos modelos enfrentan desafíos significativos al integrar conocimiento estructurado. Los problemas que presentan al alucinar y la dificultad para realizar razonamientos que involucren varios pasos de inferencia representan un obstáculo fundamental que merece una solución innovadora.

La propuesta de solución presentada por el estudiante Rolando Sánchez Ramos consiste en utilizar un modelo de lenguaje para generar una consulta en un lenguaje formal intermedio, que pueda ser ejecutada en una base de conocimientos estructurada. Posteriormente, la respuesta generada representa los datos que corresponden a la consulta inicial en lenguaje natural. Esta aproximación innovadora y técnica, plantea una solución prometedora para integrar eficazmente el conocimiento estructurado con los modelos de lenguajes, superando así los desafíos mencionados.

Es importante destacar el esfuerzo y la capacidad de Rolando Sánchez Ramos para abordar de manera independiente un tema de gran relevancia y complejidad en el área de la Ciencia de la Computación. Su habilidad técnica para implementar algoritmos que involucren sistemas de \textit{Machine Learning} y sistemas tradicionales de bases de datos es impresionante y demuestra un alto nivel de conocimiento y destreza en este campo.

Esperamos que la tesis de Rolando Sánchez Ramos reciba la evaluación máxima que merece, ya que constituye una contribución significativa al campo. Agradecemos al estudiante por su arduo trabajo y por su valioso aporte a la comunidad académica en el área de la Ciencia de la Computación.

\end{opinion}
\begin{abstract}
	Esta tesis se centra en abordar la complejidad inherente a la consulta de bases de datos en forma de grafo, como Neo4J. Estas bases de datos a menudo requieren un conocimiento especializado en lenguajes de consulta, lo que limita su accesibilidad a un grupo reducido de usuarios con habilidades técnicas avanzadas. Para superar esta limitación, proponemos la aplicación del aprendizaje \textit{Zero-Shot}, un enfoque innovador en el procesamiento del lenguaje natural. En esta investigación, se lleva a cabo un experimento basado en el modelo \texttt{GPT-4} para traducir consultas de lenguaje natural a código \textit{Cypher}. La evaluación se realiza utilizando el conjunto de datos de evaluación \texttt{MetaQA}, que abarca una amplia variedad de ejemplos de consultas. Los resultados obtenidos fueron del $76.53\%$, $43.45\%$ y $31.03\%$ para los tres lotes de evaluación del \textit{benchmark} utilizado, mejorando de esta forma el mejor resultado de modelos de lenguaje en la traducción de lenguaje natural a código \textit{Cypher} sobre \texttt{MetaQA} mediante el aprendizaje \textit{Zero-Shot}.
\end{abstract}

\begin{enabstract}
	This thesis focuses on addressing the inherent complexity of querying graph databases, such as Neo4J. These databases often require specialized knowledge in query languages, limiting their accessibility to a small group of users with advanced technical skills. To overcome this limitation, we propose the application of Zero-Shot learning, an innovative approach in natural language processing. In this research, an experiment is conducted based on the GPT-4 model to translate natural language queries into Cypher code. The evaluation is carried out using the MetaQA evaluation dataset, which covers a wide variety of query examples. The results obtained were $76.53\%$, $43.45\%$, and $31.03\%$ for the three evaluation lots of the benchmark used, thereby improving the best result of language models in translating natural language into Cypher code using Zero-Shot learning.
\end{enabstract}
\include{FrontMatter/Contents}

\mainmatter	
%===================================================================================
% Chapter: Introduction
%===================================================================================
\chapter*{Introducción}\label{chapter:introduction}
\addcontentsline{toc}{chapter}{Introducción}
%===================================================================================

\qquad 

En la época actual, asistimos a un constante aumento en la producción de información en diversos formatos: visual, auditivo y textual, que abarca todos los ámbitos de la sociedad \cite{datagenworld}. De manera particular, resulta sumamente intrigante la información generada a través del ingenio creativo y la investigación humana. Estos tipos de datos se almacenan debido a su relevancia y a la necesidad de acceder a ellos en el futuro, pudiendo optar por una organización estructurada o no. Sorprendentemente, solo alrededor del 20\% de la información a nivel mundial se encuentra estructurada \cite{structdata}.

Las bases de conocimiento constituyen un tipo particular de bases de datos diseñadas para la administración del saber. Estas bases brindan los medios para recolectar, organizar y recuperar digitalmente un conjunto de conocimientos, ideas, conceptos o datos \cite{orgkb}. La ventaja fundamental de mantener la información de manera estructurada radica en su facilidad para ser consultada, ampliada y modificada. Debido a su utilidad y prevalencia, la recuperación de información a través de consultas en bases de conocimiento se ha convertido en una tarea esencial.

Es esencial que la información almacenada en bases de conocimiento adopte un formato adecuado para permitir búsquedas ágiles y precisas. Entre los formatos más comunes se encuentran los modelos de Entidad-Relación y el modelo Relacional. A pesar de ser enfoques más antiguos, el modelo Relacional (BDR) sigue siendo el más ampliamente utilizado en la actualidad \cite{datamodel}. No obstante, en ocasiones, las características específicas del problema demandan un formato más expresivo, y es en este punto donde las bases de datos orientadas a grafos (BDOG) \cite{graphdbs} entran en juego.

Las BDOG han ganado progresivamente popularidad como una manera efectiva de almacenar información en los últimos años. Estas bases tienen la capacidad de modelar una diversidad de situaciones del mundo real al tiempo que mantienen un alto nivel de simplicidad y legibilidad para los seres humanos. Las BDOG presentan numerosas ventajas en comparación con las bases de datos relacionales. Esto incluye un mejor rendimiento, permitiendo el manejo más rápido y eficaz de grandes volúmenes de datos relacionados; flexibilidad, ya que la teoría de grafos en la que se basan las BDOG permite abordar diversos problemas y encontrar soluciones óptimas; y escalabilidad, ya que las bases de datos orientadas a grafos permiten una escalabilidad eficaz al facilitar la incorporación de nuevos nodos y relaciones entre ellos. Ejemplo de un sistema de gestión de BDOG es \textit{Neo4J} \cite{neo4j}, a través del cual es posible construir instancias de este tipo de base de datos e interactuar con las mismas a través del lenguaje de programación \textit{Cypher} \cite{cypher}, el cual posee una sintaxis declarativa similar a \textit{SQL} \cite{sqllang}.

Por otro lado, el avance en la comprensión del lenguaje natural se ha visto potenciado con el surgimiento de los grandes modelos de lenguajes (LLMs) \cite{llms} como GPT-4 \cite{gpt4} o LLaMA-2 \cite{llama2}, los cuales presentan una serie de habilidades emergentes como elaboración de resúmenes de textos, generación de código, razonamiento lógico, traducción lingüística entre otras \cite{llmsskills}. Dichas herramientas constituyen modelos de \textit{Machine Learning} entrenados con un gran volumen de datos, lo cual es posible gracias al número de parámetros con los que estos son configurados \cite{llmsparams}. 

Usualmente, para el uso de los LLMs basta con ofrecerles como dato de entrada un texto (\textit{prompt}), el cual describe la tarea que se espera que estos realicen. Además, son muchas las técnicas existentes para elaborar una entrada de calidad, esto con el objetivo de que la respuesta por parte de dicho modelo de lenguage ofrezca resultados alentadores al respecto, lo cual se conoce como \textit{prompt engineering} \cite{prompengineering}. Una técnica bastánte común es \textit{Zero-Shot Learning} (ZSL) \cite{zeroshotlearning}, la cual consiste en describirle a un LLM un procedimiento a realizar sin ofrecer de antemano ejemplos de cómo resolverlo, como por ejemplo, en tareas relacionadas con la generación de código, donde algunos de estos son capaces de generar algoritmos expresados en un lenguaje de programación formal a partir de una sentencia o consulta en lenguaje natural sin recibir como entrada del usuario algunos ejemplos de código, o especificaciones de cómo funciona el lenguaje objetivo a generar \cite{tex2code1} \cite{text2code2}. 

En lo que respecta a la comprensión del lenguaje natural y su uso en consultas a bases de conocimiento, existen diversas vías llevadas a cabo y con resultados diversos, donde se hacen análisis sintácticos y semánticos sobre la consulta, muchas veces asistidos por diccionarios o mapas sobre la base de conocimiento en cuestión. Se usan modelos de paráfrasis como técnica de aumento de datos y finalmente Transformers \cite{transformers} o incluso LLMs para llevar de la consulta ya curada al lenguaje de consulta formal o a un lenguaje intermedio capaz de expresar a esta a alto nivel \cite{text2sql1} \cite{text2sql2} \cite{text2cypher1} \cite{text2cypher2}.

Por las razones anteriormente expuestas, resulta interesante la investigación sobre la tarea de generación de código de consulta formal a partir de una sentencia en lenguaje natural mediante el uso de LLMs, especialmente el diseño e implementación un experimento capaz de demostrar las capacidades reales de estos para dicho acometido, lo cual designará la importancia de continuar el estudio de dichas herramientas con el objetivo de mejorar los sistemas de extracción de conocimientos en BDOG.

\subsection*{Problemática}
Para utilizar el lenguaje de consulta formal \textit{Cypher} se requiere de conocimientos básicos de programación, lo cual consume cierto tiempo y esfuerzo. Esto tiene como consecuencia que, solo aquellas personas con experiencia en el uso de lenguajes de programación puedan hacer uso de la mayoría de los sistemas de almacenamiento de datos desarrollados con esta tecnología y teniendo en cuenta la necesidad de poseer un conocimiento del dominio sobre el cual está construida la base de datos a consultar. Por lo tanto, llevar a cabo una mejora en las herramientas orientadas a democratizar dicho proceso permitiría hacer más rápido y eficiente dicho proceso de consulta en cuanto a tiempo y recursos computacionales. Debido a dicha situación, se propone una experimentación basada en un LLM capaz de traducir una consulta en lenguaje natural a un código en \textit{Cypher}, donde a su vez se verifique la efectividad de este a partir de enfoques basados en ZSL, los cuales intuitivamente pueden ofrecer como resultado una cota inferior para la efectividad de sistemas desarrollados en base a dichos algoritmos de aprendizaje. Además, actualmente la implementación de sistemas de generación de código de consulta formal está principalmente orientada al lenguaje \textit{SQL}, mientras que para el lenguaje \textit{Cypher}, no existen suficiente estudios recientes que avalen la calidad de tales herramientas para dicho caso de uso. 

\subsection*{Objetivos}
Dadas las ideas anteriores, los objetivos principales del trabajo consistirá en diseñar e implementar una estrategia experimental capaz de evaluar la capacidad mínima de los LLMs para la consulta en lenguaje natural a bases de conocimiento estructuradas con independencia del dominio, para lo cual se empleará un enfoque basado en ZSL.

Para lograr los objetivos generales se trazaron los siguientes objetivos específicos:

\begin{enumerate}
	\item Estudiar el estado del arte de los modelos de Aprendizaje Automático capaces de hacer predicciones de tipo texto-a-texto.
	\item Analizar el trabajo de tesis sobre este tema anteriormente desarrollado en la facultad.
	\item Implementar un modelo de Aprendizaje Automático capaz de convertir una consulta en lenguaje natural humano a un lenguaje formal que permita obtener datos a partir de una 		base de conocimiento.
	\item Explorar las capacidades de enfoques Zero-Shot para la traducción de lenguaje natural al lenguaje Cypher.
	\item Mejorar el sistema de evaluación de resultados permitiendo que el conjunto de datos de prueba y evaluación sea lo más realista posible y con una mayor complejidad.
\end{enumerate}

\subsection*{Organización de la tesis}

[Hablar sobre la estructuracion del documento]




\chapter{Preliminares}\label{chapter: chaptername}

%\newpage
\chapter{Propuesta de Solución}\label{chapter: proposedsolution}

En el presente capítulo se abordará la metodología seguida para diseñar el experimento propuesto en este trabajo \ref{experiment_defref}. Primeramente, se expondrá un marco teórico que formaliza la definición del problema a tratar, esto con el objetivo de presentar los conocimientos base tenidos en cuenta para los enfoques probados. Luego, se detallan los primeros acercamientos desechados \ref{approaches_considered}, argumentando las deficiencias de estos a la hora de arrojar resultados consistentes para la tarea que se desea desarrollar. Finalmente, se detalla la metodología definitiva a implementar, teniendo en cuenta la experiencia obtenida de las anteriores y mostrando su robustez para el análisis experimental \cite{}.

De forma general, el componente común para cada vía de solución constituye la presencia de un Gran Modelo de Lenguaje, pues representan los modelos más recientes utilizados para la tarea en cuestión; además, ofrecen resultados alentadores para el caso de traducción a lenguaje \textit{SQL} según lo visto en la sección \ref{llm_approach}. Por lo tanto, tiene sentido probar su eficacia para traducir a código en \textit{Cypher}, ya que ambos presentan similitudes como lenguajes formales declarativos para consultar bases de datos. Dicho modelo será analizado como una ``caja negra'' capaz de hacer tareas de traducción de lenguaje natural a una consulta semánticamente equivalente en el lenguaje \textit{Cypher}.

Para cada vía de solución se deberá considerar el despliegue de un sistema de gestión de bases de datos para alguna BDOG, ya que es en este componente donde se almacenará la información a extraer por consultas en un lenguaje orientado a este tipo de almacenamiento. En el caso particular de este trabajo, se considerará el uso de \textit{Neo4J}, con el cual se puede interactuar a partir del lenguaje \textit{Cypher} ya mencionado. Por esto, es importante considerar la implementación de un módulo intermedio para interactuar con una instancia del sistema de gestión \textit{Neo4J}.

El enfoque de \textit{prompt engineering} a utilizar será ZSL, por lo tanto, los textos de entrada que se le darán al modelo para la generación de código \textit{Cypher} no contendrán ejemplos de pares de lenguaje natural con su correspondiente traducción al lenguaje de consulta objetivo. Por lo tanto, se tomarán algunas ideas experimentadas en el estado del arte para \textit{SQL} vistas en la sección \ref{}.

\section{Definición formal del problema} \label{problem_formal_definition}

\section{Acercamientos considerados} \label{unused_approaches}

\subsection{Elaboración manual de consultas de prueba} \label{handmade_eval}

\subsection{Generación de consultas de prueba sintéticas} \label{synthetic_eval}

\subsection{\textit{Benchmark} orientado a \textit{Cypher}} \label{bench_eval}

\section{Propuesta de solución diseñada} \label{designed_proposal}


\chapter{Detalles de Implementación}\label{chapter: implementation}
\chapter{Análisis Experimental}\label{chapter: experiment}

En este capítulo se presentan los marcos experimentales utilizados para evaluar la efectividad del sistema propuesto en el capítulo \ref{chapter: proposedsolution} para la traducción de una consulta en lenguaje natural al lenguaje de consulta formal \textit{Cypher}. Cada enfoque utilizado consistió en el uso de un conjunto de tuplas que contenían común una consulta en lenguaje natural de ejemplo a traducir hacia un segundo elemento correspondiente con un objetivo a medir en la traducción.

Todos los experimentos fueron ejecutados en un servidor privado virtual (\textit{VPS}) \label{used_machine} con sistema operativo \textit{Ubuntu-20.04}, memoria \textit{RAM} de 16Gb, una \textit{CPU} AMD basada en la arquitectura \textit{x86\_64}, con 8 núcleos y una velocidad de 2649.998 MHz y con un ancho de banda de 16Mb/s para la comunicación con servicios como la \textit{API} de \textit{OpenAI}.

El primer sistema de evaluación fue sobre el \textit{benchmark MetaQA} \ref{classic_metaqa}, el cual constituye el principal conjunto de datos de evaluación para la tarea \textit{Text-to-Cypher} vista en la sección \ref{problem_definition}. En este caso se utilizó la versión clásica, donde los pares de evaluación consistían en una consulta en lenguaje natural con su correspondiente respuesta en la base de datos.

\section{Evaluación sobre el \textit{benchmark} \textit{MetaQA} \cite{meta}} \label{classic_metaqa}
\textit{MetaQA} \cite{metaqa} es un conjunto de datos diseñado para la tarea de razonamiento de múltiples pasos (\textit{multi-hop}) en respuesta a preguntas. Está compuesto por entidades, relaciones y preguntas en lenguaje natural relacionadas con películas. Cada nodo en el grafo de conocimientos representa una entidad (como una película, actor o director), y las aristas representan relaciones entre las entidades. El conjunto de datos también incluye preguntas a tres niveles de complejidad (\textit{1-hop,} \textit{2-hop} y \textit{3-hop}), con cada nivel requiriendo razonamiento sobre un número creciente de aristas en la base de datos en forma de grafos analizada para responder correctamente a las preguntas. A continuación se muestra un ejemplo de la distribución de dicho conjunto de datos:

\begin{table}[h]
\centering
\begin{tabular}{|c|r|r|r|}
\hline
 & \textbf{1-hop} & \textbf{2-hop} & \textbf{3-hop} \\ \hline
\textbf{Train} & 96,106 & 118,980 & 114,196 \\ \hline
\textbf{Dev} & 9,992 & 14,872 & 14,274 \\ \hline
\textbf{Test} & 9,947 & 14,872 & 14,274 \\ \hline
\end{tabular}
\caption{Distribución de los conjuntos de datos del \textit{benchmark MetaQA}.}
\label{tab:metaqatable}
\end{table}

En este estudio solo se utilizarán los datos referentes a los conjuntos de evaluación (\texttt{Test}) para cada uno de los grupos especificados, ya que el modelo empleado es un gran modelo de lenguaje mediante la técnica \textit{Zero-Shot}, por lo que no es necesario hacer un proceso de entrenamiento al mismo para realizar la tarea en cuestión, ya que se desea analizar la capacidad de inferencia del mismo sin haber sido entrenado específicamente para esta.

Las principales métricas de evaluación utilizadas fueron el número de consultas que al ser traducidas a \textit{Cypher} y ser ejecutadas ejecutadas sobre la base de conocimiento daban una respuesta idéntica a la respuesta objetivo (\texttt{correct}), así como el porcierto de dichas consultas acertadas (\texttt{correct\%}) sobre el total de consulta (\texttt{n}) y el número de consultas que generaron un código de \textit{Cypher} compilable (\texttt{compiled}), entre otras relacionadas con los recursos consumidos para el experimento como el costo monetatrio (\texttt{cost} (USD)) y el tiempo de ejecución de la evaluación en segundos (\texttt{elapsed\_seconds}). La segunda de dichas métricas fue la utilizada para comparar el resultado del modelo empleado sobre otros resultados en el estado del arte \ref{gpt4allpaper2023}. Además, implícitamente, al evaluar la efectividad del modelo sobre las consultas de los conjuntos de evaluación de \textit{1-hop, 2-hop} y \textit{3-hop}, se evalúa la eficacia del modelo sobre consultas que requieren de una relación, dos relaciones y hasta tres relaciones de conexión respectivamente para encontrar la respuesta a la consulta.

Para la preparación del conjunto de datos se insertaron los elementos correspondientes a la base de conocimientos en una instancia de \textit{Neo4J} con ayuda del componente \texttt{DBSeeder} visto en la sección \ref{dbseeder}. Luego se tomaron los conjuntos de prueba (\textit{Test}) para \textit{1-hop, 2-hop} y \textit{3-hop} y para cada par de evaluación se ejecutó el procedimiento descrito en el listado \ref{pipeline_algorithm}.

\subsection{Resultados}

Los resultados obtenidos para cada métrica analizada para cada conjunto de evaluación se muestran en la siguiente figura:

\begin{table}[H]
\centering
\begin{tabular}{|c|c|c|c|c|c|}
\hline
 & \textbf{n} & \textbf{compiled} & \textbf{correct} & \textbf{compiled\%} & \textbf{correct\%} \\ \hline
\textbf{hop 1} & 9947 & 9947 & 7613 & 100.0 & 76.53  \\ \hline
\textbf{hop 2} & 14872 & 14872 & 6462 & 100.0 & 43.45  \\ \hline
\textbf{hop 3} & 14274 & 14274 & 4430 & 100.0 & 31.03  \\ \hline
\end{tabular}
\caption{Resultados de ejecutar \texttt{GPT-4} en los conjuntos de datos de prueba de \textit{MetaQA} para \textit{hop1, hop2} y \textit{hop3}.}
\label{tab:results1}
\end{table}

En la tabla \ref{tab:results1} se muestran los resultados de eficacia del modelo \texttt{GPT-4} para traducir consultas a \textit{Cypher} tal que puedan ser utilizadas para extraer información de la base de datos objetivo. Para aquellas consultas cuyo código de \textit{Cypher} correspondiente requería de la presencia de una relacion específica entre dos entidades en cuestión se tuvo relevante resultado del $76.53\%$ de acierto. Por otro lado, aquellas consultas que requerían de la generación de una consulta con dos y haste tres relaciones tuvieron como resultados unos discretos $43.45\%$ y $31.03\%$ respectivamente, lo que nos indica la deficiencia de este modelo para responder expresiones en lenguaje natural complejas que requieran acceder a la información de más una relación entre dos entes de la base de datos en forma de grafo. Además, resulta importante mencionar la efectividad del modelo \texttt{GPT-4} para generar código de \textit{Cypher} compilable, es decir, sin errores sintácticos ni semánticos al ser preprocesado antes de ejecutar en una base de conocimiento de tipo \textit{Neo4J}. 
 
\begin{table}[H]
\centering
\begin{tabular}{|c|c|c|c|}
\hline
 & \textbf{cost (USD)} & \textbf{elapsed\_seconds} \\ \hline
\textbf{hop 1} & 139.33 & 57587.00 \\ \hline
\textbf{hop 2} & 220.66 & 79240.58 \\ \hline
\textbf{hop 3} & 218.62 & 92191.07 \\ \hline
\end{tabular}
\caption{Costo monetario y tiempo de ejecución del experimento.}
\label{tab:results2}
\end{table}

En la tabla \ref{tab:results2} es posible ver reflejados los recursos monetarios y de tiempo consumidos por la realización del experimento en el \textit{VPS} utilizado \ref{used_machine}. Como se muestra, la ejecución del modelo \textit{GPT-4} a partir de la \textit{API} de \textit{OpenAI} resulta costoso y requiere de condiciones ideales de ejecución, como por ejemplo una conexión a \textit{Internet} estable para poder acceder a la misma.

\begin{table}[H]
\centering
\begin{tabular}{|l|l|l|l|l}
\hline
Método & 1-hop & 2-hop & 3-hop \\
\hline
SOTA  & 97.50 & 98.80 & 94.80 \\
\hline
zero-shot  & 24.75 & 6.37 & 9.72 \\
\hline
zero-shot-cot & 18.41 & 12.86 & 21.89 \\
\hline
zero-shot+graph & 91.69 & 46.82 & 19.40 \\
\hline
zero-shot-cot+graph & 86.16 & 47.36 & 19.29 \\
\hline
zero-shot+graph+change-order & 95.20 & 40.48 & 20.17 \\
\hline
zero-shot-cot+graph+change-order & 95.87 & 47.71 & 23.95 \\
\hline
zero-shot Cypher Generation  & 30.00 & 10.00 & 13.00 \\
\hline
\textbf{GPT-4 zero-shot Cypher Generation}  & \textbf{76.53} & \textbf{43.45} & \textbf{31.03} \\
\hline
one-shot Cypher Generation & 99.00 & 77.00 & 96.00 \\
\hline
\end{tabular}
\caption{Comparación de los resultados de otros modelos respecto al \textit{benchmark MetaQA}.}
\label{tab:results3}
\end{table}

La tabla \ref{tab:results3} refleja el resultado del sistema implementado comparado con otros enfoques utilizados sobre \textit{MetaQA}. En cada columna de la tabla relacionada con \textit{1-hop, 2-hop} y \textit{3-hop} se reflejan los valores porcentuales de acierto de ejecución de dichas vías de solución propuestas sobre el conjunto de evaluación (\texttt{Test}) correspondiente. La primera fila contiene el mejor resultado para cada conjunto con respecto al estado del arte, las seis filas representan el resultado de utilizar \texttt{GPT-3 (code-davinci-003)} para la tarea de extracción de información de la base de datos sin utilizar lenguaje \textit{Cypher} como paso intermedio. En las filas 8 y 10 se reflejan los resultados para \texttt{GPT-3} utilizando \textit{Cypher} como vía para extraer información de una base de datos \textit{Neo4J} utilizando los enfoques \textit{Zero-Shot} y \textit{One-Shot}. Finalmente, la fila 9 contiene los resultados referentes para cada conjunto del modelo propuesto.

De acuerdo con el estudio más reciente realizado por Guo et al. \cite{gpt4graphpaper2023}, la propuesta de sistema de traducción de esta investigación supera el mejor resultado que se tenía para la traducción de lenguaje natural a lenguaje \textit{Cypher} utilizando aprendizaje \textit{Zero-Shot} sobre el \textit{benchmark MetaQA} y donde el modelo utilizado fue \texttt{GPT-3}, sin embargo, sus capacidades de extracción de conocimiento a partir de \textit{Cypher} quedan todavía lejos de los mejores resultados del estado del arte para dicha tarea.

\section{Discusiones}

Después de aplicar \texttt{GPT-4} para traducir consultas a Cypher, es pertinente destacar tanto las fortalezas como las deficiencias del sistema. Los resultados revelan una eficiencia notable en la generación de código \textit{Cypher} compilable, alcanzando un $100$\% de éxito en todas las pruebas realizadas. Este alto grado de precisión indica que el modelo es eficaz en la creación de consultas sin errores sintácticos ni semánticos, lo que es esencial para su aplicación práctica en entornos de bases de datos como \textit{Neo4J}.

Sin embargo, a pesar de esta eficacia en la compilación, el modelo demostró limitaciones en su capacidad para generar consultas correctas a medida que aumentaba la complejidad de las relaciones entre entidades. Se observó un descenso significativo en la precisión, pasando de un $76.53$\% en consultas simples (\textit{1-hop}) a $43.45$\% y $31.03$\% en consultas más complejas (\textit{2-hop} y \textit{3-hop}). Esto sugiere que, aunque \texttt{GPT-4} es competente en la traducción de consultas sencillas, su rendimiento se reduce considerablemente con consultas que involucran múltiples relaciones entre entidades.

En cuanto a las deficiencias del sistema, se identificaron varios aspectos que no se abordaron en el estudio. Uno de los más críticos fue la incapacidad del modelo para evaluar consultas anidadas y funciones de agregación, lo cual limita su aplicabilidad en escenarios de análisis de datos más complejos. Asimismo, la ausencia de un análisis multidominio impidió una evaluación adecuada de la capacidad de generalización del modelo, un factor crucial para determinar su eficacia en diferentes contextos y bases de datos. Además, el formato de las respuestas generadas por el modelo fue bastante básico, lo que plantea un área de mejora para futuras versiones, especialmente en aplicaciones que requieren un análisis de datos más detallado y avanzado.



\backmatter
%===================================================================================
% Chapter: Conclusiones
%===================================================================================
\chapter*{Conclusiones y Recomendaciones}\label{chapter:conclusions}
\addcontentsline{toc}{chapter}{Conclusiones}


\include{BackMatter/Bibliography}
\include{BackMatter/Glossary}


\end{document}